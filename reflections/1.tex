\documentclass[12pt]{article}

\usepackage{hyperref}
\usepackage[letterpaper]{geometry}
\geometry{top=1.0in, bottom=1.0in, left=1.0in, right=1.0in}
\def\UrlBreaks{\do\/\do-}

\begin{document}

\title{Freeman - A Stakeholder Theory of the Modern Corporation}
\author{Kevin Schoonover}

\maketitle

\section{Article Summary}
In the `A Stakeholder Theory of the Modern Corporation' presentation by R.
Edward Freeman, Freeman presents the idea of Stakeholder Theory in contrast to
Milton Friedman's Shareholder Theory. Stakeholder Theory asserts that all
stakeholders hold moral claim to the corporation; therefore, any company
decision must be made with all stakeholders in mind. Shareholder Theory claims
only stockholders have moral claim on the corporation. Freeman argues that
Shareholder Theory does not encapsulate all modern realities that business face
with legal and economic requirements to consumers. Then, he explains how
Shareholder Theory fills in the gaps. The presentation's thesis is that
Stakeholder Theory supplants Shareholder Theory because of limiting selfish
profit seeking from society and accounting for economic externalities.

\section{Reflection}

I find Freeman's economic argument regarding moral hazards of pass-along costs
interesting and crucial to the theory's validity. As companies grow in scale,
their ability to effect the lives of consumers and the world radically
increases. A 2017 study from CDP found 100 companies are responsible for 71\% of
the world's industrial greenhouse gas emissions. Pedro Faria, technical director
at CDP, backs-up the claim by stating the report “pinpoints how a relatively
small set of fossil fuel producers may hold the key to systemic change on carbon
emissions,”~\cite{cdp_report}. Friedman refutes the idea of social
responsibility for climate change by claiming it introduces an undemocratic tax
on consumers to implement. However, the tax is already artificially imposed on
consumers through the pass-along costs of climate change. National Geographic
reports that in the coming decade extreme weather and air pollution health costs
may spiral to at least \$360 billion annually in the US\cite{netgeo}. The \$360
billion dollars includes the cost of people fixing their homes after hurricanes,
property value of home destroyed, and other factors the consumer directly takes
a hit from.

Moreover, communities expect social responsibility from companies because of
implicit pass-along costs. One example of community expectation is the BP oil
spill of 2010. BP expected the total cost on the Gulf Coast to be \$61.6
billion~\cite{bp_spill}. For the weeks following the spill, the news and
community admonished BP for their destruction of the local environment and
sea-based businesses. The stock price dropped more than 50\%~\cite{stock}. The
community demanded BP assist and pay for the repairs because the world believed
BP had a social responsibility to that community when BP caused the spill. The
practical example of BP aligned with Freeman's pass-along costs argument which
adds to its overall validity. An argument can be made from Friedman's
perspective that it was self serving for the business to protect its image.
However, the perspective does not explain why most of the world believed that BP
had a social and ethical responsibility in the first place. BP was responsible
to all stakeholders of the Gulf of Mexico rather than stockholders within the
company.

Tesla is contemporary example of a public company leveraging social
responsibility as a business model by battling the pass-along costs of climate
change. Tesla attempts to accelerate the renewable resource market by building
electric vehicles powered by renewable sources. Elon Musk acquired Solar City as
a commitment to the renewable goal. I personally find the idea incredibly
attractive and that is why I want to buy Tesla products. These principles have
not just captivated me, but the people around me as well. Most people I talk to
love the vision of Tesla and that is the main reason they pick Tesla products.
Maybe my circle is more ideologically driven, but I find myself acting upon the
principles of Stakeholder Theory and rewarding companies who visibly practice it
more so than companies maximizing profits.  Stockholder Theory only takes this
practice into account when dealing with eleemosynary purposes, but Telsa also
has monetary profit as an objective. The successful fusion of private business
and social responsibility reaffirms my interest in the Freeman's economic
argument and Stakeholder Theory as a whole.


\bibliographystyle{plain}
\bibliography{bib/references}

\end{document}

