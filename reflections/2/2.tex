\documentclass[12pt]{article}

\usepackage{hyperref}
\usepackage[letterpaper]{geometry}
\geometry{top=1.0in, bottom=1.0in, left=1.0in, right=1.0in}
\def\UrlBreaks{\do\/\do-}

\begin{document}

\title{Solomon - Is It Ever Right to Lie}
\author{Kevin Schoonover}

\maketitle

\section{Article Summary}
In "Is it Ever Right to Lie?", Solomon presents his argument that it is never
right to lie. He concedes at the beginning that there exists a gradation of lies
because of varying consequences of outright deception vs exaggeration. Lies are
to be considered a last resort when the truth results in a nightmare, but the
truth is always more desirable according to Solomon. Then, the common
misconception about lying in advertisement is refuted by talking about a
person's ability to make value judgements and idealization is not
lying. He argues it is never right to lie with three main points: lying is an
enormous amount of effort, lying's long-term damage outweighs short-term
benefits, and mass lying would result in overall degradation of trust. 

\section{Reflection}
I want to disagree with the statement `third and strongest reason for thinking
that it is never right to lie...' as I believe "the second reason looks beyond
the short-term benefits of lying to the longer-term damage" is actually the
strongest point if you expand its scope across all people. The third reason
mentions `What would happen if lying were generally accepted?' which falls under
the analysis of long-term damage. Solomon follows Kant in saying that society
would become disfunctional as the possibility of lying would undermine any trust
that precedes the ability to tell the truth. The Kantian line of argumentation directly stems
from the second reason of long-term damage analysis and can be used as an
dystopian example to demonstrate the point. Solomon even says in his
argumentation for the second reason that "Every lie diminishes trust" and
expanding the scope to all of humanity would directly cover Kant's argument and
more.

Moreover, the second reason outweighs the first reason as I believe the first
reason is an invalid argument. Even if a lie requires `enormous amount of effort
involved in telling a lie', does that necessarily classify something as right or
wrong? I think not. If we logically continued with the line of reasoning actions
such as social reform, government policy, and certain kinds of charity would
quantify as an `enormous amount of effort involved' in performing the action.
These objectives also exist in an `an enormously complex network of interlocking
facts'. Solomon then continues with an example of `the cost of a cover-up', but
the example falls under the line of argumentation used for reason three. `The
cost of a cover-up' analyzes the aggregate long-term cost that would result from
a lie and therefore it is always preferable to tell the truth. The long-term
cost example of reason one is another subset of reason two.

Lastly, I want to reaffirm the claim that reason two is the strongest point with
an argument made by Jordan Peterson in the book 12 Rules for Life. Peterson
states `Tell the truth or at least don't lie' because lying makes a person weak
by disuniting them from their meaning. I would argue the weakness he describes
is an intuitive feeling of human beings. If I ever lie, I can sense the
described weakness of spirit. I will concede that the feeling could be
culturally generated, but it feels like it is some deep a-priori value ethic.
Peterson uses this build up to statement `you cannot get away with warping the
structure of being' and reiterates Solomon's point that a person cannot feasibly
contend with the overarching nature of fact and truth. I believe Peterson's line
of argumentation best aligns reason one with reason two as it relates back to in
the longer term `you cannot get away with warping the structure of
being' forever. The statement again appeals to the long-term analysis of lying
in which `the structure of being' will eventually get back at someone who
lies.~\cite{peterson}

\bibliographystyle{plain}
\bibliography{bib/references}

\end{document}

