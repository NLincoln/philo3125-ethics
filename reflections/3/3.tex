\documentclass[12pt]{article}

\usepackage{hyperref}
\usepackage[letterpaper]{geometry}
\geometry{top=1.0in, bottom=1.0in, left=1.0in, right=1.0in}
\def\UrlBreaks{\do\/\do-}

\begin{document}

\title{Leiser - Truth in the Marketplace}
\author{Kevin Schoonover}

\maketitle

\section{Article Summary}
In `Truth in the Marketplace', Leiser presents cases and categories of false
advertisements for products within the market. The article claims advertisers and
manufactures have a moral obligation from their duties as citizens and human
beings to disclose the whole truth to protect consumers from the promotion of
dangerous products, false claims of effectiveness, misleading statements or
contexts, the concealment of the truth, and misleading forms of advertisement.
The advertisers should go above and beyond the law and uphold the moral law to
prevent actions like the dangerous drugs Raudixin and Commel from being
repackaged and used in Brazil. The article concludes that government regulation
should be imposed on the advertisement industry to uphold honest advertising and
make agencies and producers responsible for paying for damages to consumers.

\section{Reflection}
I agree with Leiser's overall conclusion about regulating the advertisement
industry, but disagree with the means of government enforcement. My main
contention is most government projects and enforcement of the law with these
large companies requires a significant amount of time to provide restitution.
For example, in the case of Bernie Madoff, some victims of the ponzi scheme are
still waiting to receive their restitution over seven year later~\cite{madoff}.
In the case of false advertising, waiting over seven years may not be soon
enough for the people who catch `fatal blood diseases' from Commel or `deep
depressions that hospitalization was often necessary, and suicide sometimes
followed' from Raudixin. The drug advertisements are an extreme case, but
demonstrate the need for a better solution.

A potential solution to the independent verification of foods, drugs, and other
products that could harm consumers without slow government intervention is to
build a privatized verification industry. Verification of products allow for
circumventing many of the problems Leiser has with the advertising agency. The
agency's can still conceal the truth, have misleading statements or contexts,
promote dangerous products, and the other problems mentioned; however, a
verification stamp on the product (when not doctored) can prevent the specific
harm to the consumer. The scope of verification can also be extended to include
advertisements focused on providing the `whole truth' standard. The current
standard of approval, the Food and Drug Administration (FDA), primarily airs on
the side of caution, taking upwards of a `6 month goal' to approve drug clinical
trails financed by the company~\cite{fda} while only focusing on the drug
itself and not advertisements. The FDA has a monopoly on the market within the
United States. Not only is this `6 month goal' slow for life saving medication
to be trusted, the FDA also has inherit basis towards sponsors of the FDA
approval work~\cite{bias}. The privatization of this industry would allow for
competition to maximize the safety and preferences of the consumer while
minimizing the inherit bias. If consumers feel a verification company is bias,
they no longer buy a product endorsed by it and the producer is forced to get a
new verification. Consumers would hold the companies responsible to high-quality
verification practices in a compelling way as opposed to the monopoly of the
FDA.

My main concerns of verification privatization are ensuring that another
monopoly does not form and will competition even work in that industry. For the
sake of argument, I will assume that privatization happens. It is possible that
one verification company may take the vast majority of the market share as
people only trust this one specific company as a replacement for the FDA. Their
superior testing methodology, technology, etc make them the best option and they
crush their competition. The scenario is synonymous with something like Google
today. Even if this occurs, the government still has the change to regulate the
company just as they do with other monopolies. The point of my argument is not
to disavow regulation just to say that privatization may work better in this
instance. Secondly, competition within this space would come at the cost of
lives. Early adopters of the new system will be at a much greater risk for
inferior products. The risk to early adopters can be mitigated with the FDA
producing studies in tandem that slowly validate a few key products and then
slowly supervising the market for a few years.

\bibliographystyle{plain}
\bibliography{bib/references}

\end{document}

