\documentclass[12pt]{article}

\usepackage{hyperref}
\usepackage[letterpaper]{geometry}
\geometry{top=1.0in, bottom=1.0in, left=1.0in, right=1.0in}
\def\UrlBreaks{\do\/\do-}

\begin{document}

\title{Davis - Some Paradoxes of Whistle-Blowing}
\author{Kevin Schoonover}

\maketitle

\section{Article Summary}
In Some Paradoxes of Whistle-Blowing, Davis argues that the standard theory of
whistle-blowing\footnote{The standard theory of whistle-blowing, according to
Davis, if the organization will do serious and considerable harm to the public,
the would-be whistle-blower has concluded the superior will do nothing
effective, the would-be whistle-blower has exhausted internal resolution
measures, the would-be whistle-blow has evidence that would convince an observer
that her threat is viable, and the whistle-blower believes that intervening will
prevent the harm at a reasonable cost.}is paradoxical in nature with three
primary paradoxes: the paradox of burden, the paradox of missing harm, and the
paradox of failure.  He shows how the Roger Boisjoly's Challenger disaster case
highlights these paradoxes: it permanently affected his career and life (the
paradox of burden), only indirectly prevent harm via falsification of records
(the paradox of missing harm), and his testimony did not prevent any real harm
which is typical to whistle-blowing cases (the paradox of failure). To solve
these paradoxes, Davis proposes complicity theory that emphasizes the moral
obligation for yourself to avoid moral wrongs. This presupposes moral wrongdoing
fixing paradox of missing harm, enforces an obligation to set things right if we
are engaged in some wrong fixing paradox of missing harm, and focuses on
preventing complicity in wrongdoing which can be solved by whistle-blowing
fixing the paradox of failure.

\section{Reflection}
I believe that Davis argument is very compelling for some of the more nuanced
cases of whistle-blowing, but still has a blind spot by not defining what
`serious moral wrongdoing' means. Edward Snowden
exposing the NSA PRISM plan closely parallels that of Boisjoly's case. With
regards to the standard definition, it is not obvious that the NSA was causing
serious and considerable harm to the public by collecting their phone records.
It is inconclusive whether Snowden attempted to change the program from the inside
before leaking the program to the public.  Moreover, in the wake of Snowden,
there does not appear to be any significant change that prevents the "harm" of
government surveillance in our everyday life.  The case seems to fail the
majority of Davis' standard theory of whistle-blowing; however, many people would
intuitively consider what Snowden did to be whistle-blowing. 

When taking the Snowden case under the lens of complicity theory, it seems to
fall in-line with the new definition of whistle blowing. Snowden's work as an
administrator gave him access to the documents detailing the PRISM program in
which he was voluntarily employed (C1 \& C2). In the interviews, Snowden
constantly states he thoroughly believed that the US government spying on its
own citizens was a morally wrong invasion of liberties (C3) and he was complicit
by protecting the networks containing the documents (C4). I believe complicity
theory becomes very philosophically interesting for the Snowden case for if he
is justified in his beliefs (C5) and if C3 and C4 are true(C6).

Do citizens of governments inherently have privacy or more specifically is it
morally wrong for a government to spy on its citizens? Snowden believes yes
which satisfies the justification case, but its it `true' according to C6?
Rhodri Jeffreys-Jones argues that the government has a moral obligation to spy
on its citizens to protect them. Other accounts say it is for the most part
ethical when used for the purpose of protection with the consent of the people
and is not abused by trusted insiders or third
parties\footnote{https://cs.stanford.edu/people/eroberts/cs181/projects/ethics-of-surveillance/ethics.html}.
Lastly, some say it is morally wrong as it violates the privacy of the
individual\footnote{https://www.iep.utm.edu/surv-eth/}, but why privacy is
valuable is outside the scope of this discussion. While I do not necessarily buy
all the arguments, it does highlight where the `serious moral wrongdoing'
portion of the theory without being chained to some theory of wrongdoing is not
easily classifiable. I believe that the Snowden case is whistle-blowing, but it
is not perfectly clear under complicity theory under this condition of `serious
moral wrongdoing'.

\end{document}

