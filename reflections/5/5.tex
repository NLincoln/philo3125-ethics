\documentclass[12pt]{article}

\usepackage{hyperref}
\usepackage[letterpaper]{geometry}
\geometry{top=1.0in, bottom=1.0in, left=1.0in, right=1.0in}
\def\UrlBreaks{\do\/\do-}

\begin{document}

\title{Duska - Whistle-Blowing and Employee Loyalty}
\author{Kevin Schoonover}

\maketitle

\section{Article Summary}
In Whistle-Blowing and Employee Loyalty, Duska argues that an employee has no
obligation to be loyal to their company and therefore whistle-blowing is
permissible. Duska believes that one cannot be loyal to a company as it gives
companies moral status that fails to consider the difference between people and
corporations. The object of loyalty is ordinarily to a person or a group of
people. Duska makes the point that a corporation is not a person of itself, but
an instrument. To prove this point, Duska shows distinctions between companies
and two groups normally considered to be objects of loyalty: family and teams.
In the family case, the binds that tie a corporation together is not sufficient
to be an object of loyalty like in a family. If an employee would undermine or
diminish the corporate goal, the corporation owes the employee no loyalty and
should fire them to help fulfill the company's goal.  Duska continues with the
distinction between business and a sports team by claiming business is not a
standard game.  Business is not controlled by some referee, the consequences of
businesses affect everyone, and the match is infinite in length, therefore, is
not comparable.

\section{Reflection}
Duska's account of whistle-blowing permissibility seems largely incomplete. Duska
makes an important exception when stating `If the company can get its employees
to view it as a team they belong to, it is easier to demand loyalty. Then the
rules governing teamwork and team loyalty will apply.' The argument regarding
business hinges on business being inherently selfish with the goal of dividing
labor or making a profit. While I think these are both goals of some companies,
over time I have noticed companies that dominate the market are not necessarily
profit or division of labor driven but goal driven. Companies like Tesla, Space
X, etc. whether right or wrong have been aligning themselves with specific
visions of the future to push the status quo by making great products. I would
argue that these companies by forgoing competition and focusing purely on their
own products and markets fall under the exception Duska mentions.

In particular with Tesla and SpaceX, Elon Musk has often talked about how he saw
an idea of the future with going to Mars and electric cars and decided to make
Tesla and SpaceX. He said in a Tweet,
\footnote{https://twitter.com/elonmusk/status/976414452106055680?lang=en} `If it
was about money, I’d just do another Internet company.' I will assume for the
sake of argument based on anecdotal evidence most people who work for Elon put
in long hours for relatively low pay because they ascribe to this team value and
overall mission. Now, let us say that someone in the company determines that
Elon Musk was secretly using uranium to power the rockets that were giving
workers cancer, but we only have 3 years for humanity to survive on earth.
According to this account, the loyalty to the public would be in contest. On one
hand, the rockets are secretly killing the workers, but on the other hand, the
rockets could save humanity. This is a relatively contrived example, but I think
in this case loyalty to the mission and the company could be one and the same
which would make Duska's argument weak. In a talk by Simon
Sinek\footnote{\url{https://www.youtube.com/watch?v=_osKgFwKoDQ}}, Sinek comes
to the same conclusion in which the values of a company allow the company to
play an infinite game and the employees of that company align themselves with
those values breeding loyalty.

The line of argumentation I present relies on the jump that aligning one's
values to a company breeding loyalty is equivalent to having loyalty to the
company, but I think this is justified. When I talk to my friends about products
they buy I often hear, `I won't buy from X company because they have bad
business practices' or `I only buy from X because I believe in their mission`.
Maybe I have ideologically driven friends, but when you look at brand loyalty
that people have to an iPhone or Mac computers it seems self-evident. People
external to the companies have loyalty to the company themselves and will only
buy for example Apple products. I am a Tesla fanboy and have loyalty to the
Tesla products, marketing, and business practices. It does not seem like too far
of a jump to say that if someone works at a company they too experience this
same pride for the vision and in turn have a loyalty to the vision and company
forming the team dynamic exception.

\end{document}

